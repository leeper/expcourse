\documentclass[12pt,a4]{article}
\usepackage[top=.75in, bottom=.75in, left=.75in, right=.75in]{geometry}
\usepackage{titlesec}

\titleformat*{\subsection}{\large\it\bfseries}
\titlespacing*{\section}{0pt}{6ex}{0ex}

\begin{document}

\thispagestyle{empty}
\noindent{\LARGE\textbf{Practical and Ethical Scenarios}}

Imagine yourself as the researcher in each of the following scenarios. What is ethical, responsible, and methodologically necessary in each situation? {Note: There may not be a right answer.}

\begin{enumerate}\itemsep1em

\item Benefit/harm trade-offs

\item The researcher wants to know if mosquito nets help reduce the spread of malaria in a Sub-Saharan Africa country, which prior research suggests are effective but the size of the effect is ambiguous. Some regions in the country have many malaria-carrying mosquitos, while some regions have few. The researcher randomly assigns households across all regions to receive mosquito nets or not. % ethics of randomization

\item The researcher is interested in whether giving grants to local libraries for community outreach initiatives increases use of the libraries. Libraries are randomly assigned to receive either no grant, a small grant (20,000 kr), or a large grant (100,000 kr). After the study is over, the researcher learns that due to an administrative error, several libraries that were supposed to receive large grants received no grants, and vice versa.

\item All details are the same as the previous scenario, but the ``administrative error'' was made intentionally by the spouse of a local library administrator.

\item The researcher is interested in whether international study experiences by high school students affects the completion time for their post-secondary degrees (either positively or negatively). One thousand students at a university are invited to participate in the study, which will use a lottery to allocate scholarships for stays abroad. Twenty of the students who do not receive a scholarship in the lottery apply elsewhere for a scholarship and eventually study go abroad.

\item Deception: Ommission

\item Deception: Commission

\item The researcher's goal is to study the effects of bribe-taking on local governments' responsiveness to citizen concerns. The researcher randomly assigns municipalities to be in ``bribe'' or ``no bribe'' conditions, surveys residents in each municipality about their concerns, encourages anyone with a concern to contact a local official. For those living in ``bribe'' municipalities, the researcher also supplies a modest sum of money and instructs the residents to offer the bribe.

\item Attrition

\item Informed consent

\item Children/youth

\item Prisoners

\item Cancer patients

\item The researcher is interested in whether ticket enforcement on municipal buses is cost effective (i.e., whether the cost of paying ticket enforcers is outweighed by the gain from more riders paying for their tickets). The researcher believes that ticket violations (i.e., not buying a ticket) are highest in the city's ten districts with the lowest median income. The researcher randomly assigns five of these districts to receive increased enforcement. %Differential enforcement across regions; generalizability to other regions

\item The researcher wants to know if leadership training for ministerial managers improves the productivity of their employees. The training program's effectiveness will be assessed every six months and continue for three years. After 12 months the program shows a modest, positive effect but it is hard to distinguish from no gain. After 18 months, the effect appears to be smaller but still positive. The ministry receives additional funding for the program and suggests to the researcher that the study be discontinued and the program expanded to all managers.

\item Discontinuing a study early: apparent ineffectiveness

\item Pretreatment

\item The researcher is interested in whether a new primary school math curriculum increases student math performance. Schools are selected for participation in the study and teachers are randomly assigned to either the new curriculum or the old curriculum to use in the next school year. While happy to participate in the study, after receiving their assignments several teachers express concerns that being assigned to their less preferred curriculum means they won't teach as well as they otherwise would. % treatment preferences

\item The researcher is interested in whether the time of day that hospital doctors work (morning, afternoon, evening) affects affects medical error rates. Doctors are told they will participate in an experiment where their working hours are randomized. Doctors are asked for their preferred working shift. Forty percent say they do not have a preference, while sixty percent have a strong preferences for one of the three options. The researcher uses only those expressing no preference in the study. % treatment preferences

\item The researcher's goal is to study % Subject pool contamination

\item Interference/SUTVA violations

\end{enumerate}

\end{document}