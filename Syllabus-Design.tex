\documentclass[12pt,a4paper]{article}
\usepackage[top=1in, bottom=1in, left=1in, right=1in]{geometry}
\usepackage{graphicx,setspace,hyperref,amsmath,amsfonts,multirow,ccaption,mdwlist,comment}
% mini table of contents
\usepackage{minitoc}
\dosecttoc % make section toc
\setcounter{secttocdepth}{2} % subsection depth
\renewcommand{\stctitle}{} % no title
\nostcpagenumbers

% optionally include commented environments
\excludecomment{lessonplan}
\excludecomment{finalexam}

\setlength{\marginparwidth}{.5in}
\usepackage{natbib}
% Two lines to create in-text full citations for a syllabus
% And comment out my other standard bibtext commands
\usepackage{bibentry}
\newcommand{\reading}[2][]{\noindent --{#1} \bibentry{#2}.\vspace{.25em}\\}
\newcommand{\seealso}{\noindent \emph{See Also:}\\}
\newcommand{\topic}[1]{\noindent \textbf{#1}\\}
\usepackage[T1]{fontenc}
\usepackage{lmodern}
\hypersetup{
    bookmarks=true,         % show bookmarks bar?
    unicode=false,          % non-Latin characters in Acrobat’s bookmarks
    pdftoolbar=true,        % show Acrobat’s toolbar?
    pdfmenubar=true,        % show Acrobat’s menu?
    pdffitwindow=false,     % window fit to page when opened
    pdfstartview={FitH},    % fits the width of the page to the window
    pdftitle={Syllabus: Experimentation},    % title
    pdfauthor={Thomas J. Leeper},     % author
    pdfsubject={Political Science},   % subject of the document
%    pdfkeywords={politics} {public opinion} {information acquisition} {media exposure} {time}, % list of keywords
    pdfnewwindow=true,      % links in new window
    pdfborder={0 0 0}
}

\title{Experimentation and Causal Inference}
\author{Thomas J. Leeper\\
Department of Political Science and Government\\
Aarhus University}

\begin{document}
\nobibliography*

\maketitle

\faketableofcontents

%\section{Introduction}

The purpose of this course is to introduce and elaborate identification-oriented research methods, particularly experimentation. While this course could be taught in a number of different ways, the focus here is on delivering a breadth substantive research topics and methodological considerations that emerge in experimentation. The course also touches (to a much lesser extent) on quasi-experimental research methods for causal inference and is bookended by discussions about the nature of causation and alternative means of inferring causal relationships.

There is an effort to touch on analysis of experiments, but this material is less central than consideration of causation per se, experimental design (as opposed to analysis), and the substantive topics that might be studied experimentally. A short unit on analysis is presented in the middle of the course. This, too, is done somewhat atypically. Rather than spending large portions of time on particular estimation techniques (i.e., difference-of-means, ANOVA, regression, permutation) the emphasis is on analytic considerations with some discussion of particular techniques. Those interested in particular techniques can reference any of the recommended textbook reading listed at the end of this section, consult with the instructor, or examine the large number of R packages for experimental inference available at \url{http://cran.r-project.org}.

One should leave this class with a deep and broad understanding of experimental design, along its challenges and opportunities. You should be prepared to design your own experiment and have developed an ability to read experimental literature, as well as critique causal claims more broadly.

\subsection*{Books}
The assigned texts for this course are:

\reading{ShadishCookCampbell2001}
\reading{GerberGreen2012}

\noindent As a social researcher interested in causation, you may also find the following books help:\\

\reading{Rubin2006}
\reading{MorganWinship2007}
\reading{Rosenbaum2009}
\reading{GelmanHill2006}
\reading{AngristPischke2008}
\reading{FieldHole2003} % primer
\reading{MortonWilliams2010}
\reading{Mutz2011}
\reading{Druckmanetal2011}


\section*{Expectations and Evaluation}
As should be obvious, students are expected to have read all material before the day on which it is assigned. Class meetings will be discussion focused, with lectures occurring fairly rarely. Participation during in-class discussions will be the basis for the plurality of a student's grade. If any student is uncomfortable with this arrangement, please consult with the instructor immediately at the beginning of the term.

Each student is also expected to complete four short, one-page (double-spaced) response papers that concisely respond to key issues raised by readings on a given week. These will serve to catalyze discussion each week and should be distributed to the entire class 24-hours in advance of classtime. A schedule for who will write papers when will be set during the first class meeting.

Finally, each student is expected to complete a research design paper, which will be due the final week of the course. This paper should outline an important social research question, apply relevant theory and literature to that question, derive hypotheses from that theory, and propose an experimental (or quasi-experimental) research design to test those hypotheses. The paper should address any relevant issues raised during the course, such as statistical power, analysis techniques, compliance, ethics, etc. Students should contact the instructor well in advance of the deadline to discuss topics and plans for the paper. Students should aim for 15 double-spaced pages (+/- 3 pages).\\

\begin{comment}
\textbf{Evaluation Formula}
\begin{itemize*}
\item 40\% In-class participation
\item 20\% Four response papers (5\% each)
\item 40\% Research design paper (due last week of class)
\end{itemize*}
\end{comment}

\begin{finalexam}
\clearpage
\section{Exam}
The exam will consist of a written report planning the design, execution and analysis of an experimental test of a clearly defined empirical problem.


\end{finalexam}



\section*{First 2/3 of Course: Experimental Design, Analysis, and Applications}

\subsection*{History and Traditions}\label{sec:traditions}
\reading{Druckmanetal2006}
\reading{Gosnell1926}
\reading{Hovland1959}
\reading{McDermott2002}
\reading{Danziger2000}

\seealso
\reading{MortonWilliams2008}
\reading{Iyengar2011}
\reading{Gerber2011}

\subsection*{Causal Inference}\label{sec:causation}
\reading{Holland1986}
\reading{ShadishSullivan2010}
\reading{Rubin1978}
\reading{Manski1999}
% probably something from Freedman

\seealso
\reading{Splawa-Neyman1990}


\subsection*{Validity}\label{sec:validity}
\topic{Concepts and Construct Validity}
\reading{Goertz2005}
\reading[Selection from]{ShadishCookCampbell2001}
\reading{Pitkin1967}
\reading{AdcockCollier2001}

\topic{Internal Validity}
\reading{McDermott2011}
\reading[Selections from]{ShadishCookCampbell2001}

\topic{External Validity}
\reading[Selections from]{ShadishCookCampbell2001}
\reading{Cronbach1986}
\reading{Sears1986}
\reading{DruckmanKam2011}


\subsection*{Experimental Analysis}\label{sec:analysis1}
\topic{Estimands and Estimates}
\reading[Selections from]{GerberGreen2012}
\reading{ManskiNagin2002}
\reading{Gill1999}
% Randomization-based methods
% Difference of means, medians, and variances
% ANOVA
% Regression
\seealso
\reading{Manski1990}

\topic{Research Synthesis}
\reading{MoherDulbergWells1994}
\reading{Bloom1995}
% something from Hedges HB
\reading{GerberGreenNickerson2001}
\reading{LauSigelmanRovner2007}
\reading{AnsolabehereIyengarSimon1999}

\subsection*{More Analysis}\label{sec:analysis2}
\reading{GainesKuklinski2011a}
% probably something from Gerber and Green
\reading{AnsolabehereRoddenSnyder2008}

\topic{Noncompliance}
% distinction between ITT, ATE, ATT/LATE
\reading{AngristImbensRubin1996}
\reading[Selections from]{GerberGreen2012}

\topic{Mediation}
\reading{BaronKenny1986}
\reading{Imaietal2011}
\reading{BullockHa2011}

\topic{Covariates/post-stratification/blocking}
\reading{Bowers2011}

\subsection*{Ethics}\label{sec:ethics}
\reading{APSAEthics}
\reading{NAS1995}
% add something about funding?

\topic{Human Subjects and Harm-Benefit Tradeoffs}
\reading{BelmontReport}
\reading{SingerLevine2003}
\reading{Kunda1990}
\reading{ButlerBroockman2011} % elites as subjects

\seealso
\reading{Zimbardo1973}
\reading{Brandt1978}
-- Nuremberg Code. \url{http://ohsr.od.nih.gov/guidelines/nuremberg.html}\\

\topic{Deception}
\reading{Milgram1963}
\reading{Baumrind1964}
\reading{Kelman1967}
\reading{Baumrind1985}
\reading{Herrera2001}
\reading{HertwigOrtmann2008}

\topic{Randomization}
\reading{CorriganSalzer2003}
\reading{BottiIyengar2006}
% possibly article about 'equipoise'
\reading{KuklinskiQuirk2011b}

\subsection*{Lab Experiments: Applications}\label{sec:lab}

\topic{Psychology-style Experiments}
\reading{IyengarKinder1989}
\reading{Mutz2007}

\topic{Economics-style Experiments}
\reading{Habyarimanaetal2007}
\reading{Palfrey2009}
\reading{Miller2011}
\reading{ColemanOstrom2011}

\topic{Group and Deliberation Experiments}
\reading{KarpowitzMendelberg2011}
\reading{Druckman2004a}

\subsection*{Survey Experiments: Applications}\label{sec:survey}
\reading{Sniderman2011}
\reading{GainesKuklinskiQuirk2007}
% Something about web experiments
% maybe something about MTurk

\seealso
\reading{BlairImai2012}

\subsection*{Field Experiments: Applications}\label{sec:field}
\reading{GerberGreenLarimer2008}
\reading{Sondheimer2011}
\reading{Sinclair2011}
% something on cluster-randomized experiments
\reading{Gerberetal2011}


% Maybe something on reporting of experiments
%% talk about graphics
%% Read Riker (APSR, 1967) for his conversation discussion of an experiment
%% Something else on reporting standards


\section*{Next 1/4 of Course: Nonexperimental Routes to Causal Inference}

\subsection*{Matching}\label{sec:matching}
\reading{Rubin2008}
\reading{Sekhon2009}
\reading{RosenbaumRubin1983}
% other readings from Rubin book
\reading{IacusKingPorro2009}
\reading{ArceneauxGerberGreen2010}
\reading{LaLonde1986}
\reading{HoImaiKingStuart2007}
\reading{Steineretal2010}

\topic{Dose-Response}
\reading{Imbens2000}
\reading{ImaivanDyk2004}
\reading[Section 4.6.3 of]{MorganWinship2007}

\topic{Immutable Characteristics}
\reading{BoydEpsteinMartin2010}
\reading{GreinerRubin2011}

\subsection*{Real World Randomization and Approximate Randomization}\label{sec:natural}
\reading{SekhonTitiunik2012}
\reading{Angrist1990}
\reading{EriksonStoker2011}
\reading{ChattopadhyayDuflo2004}
\reading{HoImai2008}

\subsection*{Discontinuities}\label{sec:discontinuities}
\topic{Time}
\reading{CampbellRoss1968}
\reading{BertrandDufloMullainathan2004}

\topic{Geography}
\reading{CardKrueger1994}
\reading{DellaVignaKaplan2007}
\reading{HuberArceneaux2007}
% Keele paper on geographical boundaries

\topic{Other}
\reading{AngristLavy1999}
\reading{HainmuellerKern2008}
\reading[Selections from]{ShadishCookCampbell2001}
\reading{ImbensLemieux2008}
\reading{Bloom2009}


\section*{Final 1/12 of Course: Controversy of Approaches}\label{sec:controversy}
\reading{MahoneyGoertz2006} % two cultures article
\reading{ImaiKingStuart2008}
\reading{KingKeohaneVerba1994}
\reading{BradyCollier2010}
\reading{Sekhon2004}


% load bibtext, but don't generate a bibliography
\bibliographystyle{plain}
\nobibliography{Syllabi}

\end{document}